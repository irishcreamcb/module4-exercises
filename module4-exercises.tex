\documentclass{article}
\usepackage{amsmath}
\usepackage{amsthm,amssymb}
\usepackage[a4paper,left=25mm,right=25mm,top=30mm,bottom=30mm]{geometry}
\usepackage{fancyhdr}
\usepackage{titlesec}
\usepackage{enumerate} 
\usepackage{graphicx}
\usepackage[dvipsnames]{xcolor}
\usepackage{tikz} 
\usepackage{cancel}
\usepackage{parskip}
\usepackage[condensed,light,math]{iwona}
\usepackage[T1]{fontenc}
\usepackage{booktabs}

\title{probability exercises}
\author{emilianna louise abundo limlengco} 
\date{\today} 

\makeatletter % yoinked from https://tex.stackexchange.com/questions/14071/how-can-i-increase-the-line-spacing-in-a-matrix
\renewcommand*\env@matrix[1][\arraystretch]{%
  \edef\arraystretch{#1}%
  \hskip -\arraycolsep%
  \let\@ifnextchar\new@ifnextchar%
  \array{*\c@MaxMatrixCols c}}
\makeatother


\fboxsep=4pt
\RenewDocumentCommand{\footnoterule}{}{\vfill\kern-3pt \hrule width 0.4\columnwidth\kern2.6pt} %yoinked from LSE

\RenewDocumentCommand{\labelitemi}{}{$\rightarrow$}
\RenewDocumentCommand{\labelenumi}{}{\colorbox{pink}{\textbf{\arabic{enumi}}}}
\RenewDocumentCommand{\labelenumii}{}{\colorbox{CornflowerBlue!50}{\textbf{\alph{enumii}}}}

\NewDocumentEnvironment{solution}{}{%
    \RenewDocumentCommand{\qedsymbol}{}{$\blacksquare$}
    \begin{proof}[Solution]
}{\end{proof}}

\makeatletter
\newcommand*{\defeq}{\mathrel{\rlap{%
                     \raisebox{0.3ex}{$\m@th\cdot$}}%
                     \raisebox{-0.3ex}{$\m@th\cdot$}}%
                     =}
\makeatother

\begin{document} 

\section*{How to Use this Reviewer}
Hello! This is a compilation of solved exercises for module 4 of MATH 51.4. All of these exercises are taken straight from Aldrich and Cisco's course notes, so you can expect tests 
to be very similar to the items given. Some items are bound to be a little bit trickier than others, so I'll note when these items show up.\par Normal items will look like this:\begin{enumerate} 
    \item A very easy math problem. What's 1 + 1?
\end{enumerate} 
whereas difficult problems will be soulless, like this:\begin{enumerate}\setcounter{enumi}{1}
    \RenewDocumentCommand{\labelenumi}{}{\fcolorbox{magenta}{white}{\textbf{\arabic{enumi}}}}
    \item A very difficult math problem. Prove that $\displaystyle \binom{2n}{n} < 2^{2n-2},~\forall n \geq 5$ using induction. 
\end{enumerate} I might also include warnings in my \textbf{Nerd Interjections!}\par
\parindent=25pt \begin{minipage}[t]{.14\textwidth}
    \vspace{0pt}
    \includegraphics[width=2cm]{nerd_maddy.png}
\end{minipage}%
\fbox{
\begin{minipage}[t]{.76\textwidth}
    \vspace{0pt}
    \textbf{Nerd Interjection!}\footnote{Image from @Ellem\_\_ on Twitter.} These sections are for me to remind you of some necessary information to solve the problems, elaborate on 
    something that I think isn't all that clear with just pure math symbols, give a helpful theorem, be an annoying piece of shit, anything, really! Just think of it as a tips and tricks section. 
\end{minipage}%
}\parindent=0pt \par I also have another section called \textbf{Can we Prove it?} (unfortunately lacking a cute picture to go along with it; Mikh 
was nice enough to edit one up for me, but I haven't been able to format it in a way I like), where I include some interesting, not really necessary, but 
nonetheless relevant proofs. So far, these two are my only two gimmicks, but I might add more in the future.\par
\fbox{\begin{minipage}[t]{0.98\textwidth}
    \vspace{0pt} 
    \textbf{Can we Prove it?} This is just a random proof I yoinked from our homeworks.\begin{proof} 
        ($ \implies $) Let $ x \in (A \cap B) \setminus C $. Then, $ x \in (A \cap B)$ and $ x \notin C $. \\
        \phantom{($ \implies $)} Since $x \in (A \cap B)$, $ x \in A$ and $ x \in B$. \\
        \phantom{($ \implies $)} Since $x \in A$ and $x \notin C$, $x \in (A \setminus C) $. \\
        \phantom{($ \implies $)} Since $x \in B$ and $x \notin C$, $x \in (B \setminus C) $. \\
        \phantom{($ \implies $)} Thus, $x \in (A \setminus C) \cap (B \setminus C) $. \\ 
        \\
        ($ \impliedby $) Let $ x \in (A \setminus C) \cap (B \setminus C) $. Then, $ x \in (A \setminus C) $ and $ x \in (B \setminus C) $. \\ 
        \phantom{($ \impliedby $)} Since $ x \in (A \setminus C) $, $ x \in A $ and $ x \notin C $. \\
        \phantom{($ \impliedby $)} Since $ x \in (B \setminus C) $, $ x \in B $ and $ x \notin C $. \\
        \phantom{($ \impliedby $)} Since $ x \in A $ and $ x \in B $, $ x \in (A \cap B) $. \\
        \phantom{($ \impliedby $)} Thus, $ x \in (A \cap B) \setminus C $. \\
        \\ 
        Since both sides of the conditional are true, it holds that $ (A \cap B) \setminus C = (A \setminus C) \cap (B \setminus C) $. 
    \end{proof} 
\end{minipage}%
}\par
Finally, there are blue boxes to indicate when instructions aren't obvious from the question itself, or if there are similar items that can be grouped together.\par
\parindent=25pt 
    \colorbox{CornflowerBlue!50}{
    \begin{minipage}[c]{0.9\textwidth}
        \centering
        For items \#7 to \#12, we need to reevaluate our life decisions.
    \end{minipage}
    }\parindent=0pt \par 
It's very important to note that this is a \textit{work in progress!} I am human, and I will make mistakes, and I cannot finish doing all the exercises within the span of one day. If you spot anything wrong, 
please feel free to message me; I will correct it as soon as possible.\par
As a final note, these are not replacements for the modules/paying attention in class, these are supplements for them. I won't explain all the topics here, and I'll assume that you at least have 
read the basics, so don't treat these reviewers as your only source of information. Our teachers spends a lot of time on the handouts, they're really good! (except when they're wrong) With that, though, I think 
I've covered all pertinent points. Good luck, and happy studying!
\pagebreak 

\section*{4.1: Determinants }
\textit{This is mostly pure computation, but there are some proving and show questions later on, which are a little bit interesting. Also, spoiler, items \#10 and \#11 are foreshadowing for 
eigenvalues. You'll see!}

\begin{center}
\colorbox{CornflowerBlue!50}{
    \begin{minipage}[c]{0.9\textwidth}
        \centering
        Find the determinants of the following matrices. In case they have elements with variables, solve for the determinants in terms of those variables. 
    \end{minipage}
}\end{center}
\begin{enumerate}
    \item \(\begin{bmatrix}
        1&2\\3&4
    \end{bmatrix}\)\begin{solution}
        We can just use the basic \(ad-bc\) formula for a \(2\times 2\) matrix, giving us \(4 - 6 = -2\). 
    \end{solution}
    \item \(\begin{bmatrix}
        0&6&0\\3&99&-1\\-2&99&5
    \end{bmatrix}\)\begin{solution}
        We pick the row with the most zeros, then use the cofactor expansion.\[
            \begin{vmatrix}
                0&6&0\\3&99&-1\\-2&99&5
            \end{vmatrix} = 0 \begin{vmatrix}
                99&-1 \\ 99&5
            \end{vmatrix} - 6 \begin{vmatrix}
                3 & -1 \\ -2 & 5 
            \end{vmatrix} + 0 \begin{vmatrix}
                3 & 99 \\ -2 & 99 
            \end{vmatrix} = -6 \,(15 - 2) = -78. 
        \] You can notice that because of where these 99's and 0's are placed, it doesn't matter what row or column we pick; we're never going to have to multiply a 99 by something. Try it out for yourself and 
        pick a different option!
    \end{solution}
    \item \(\begin{bmatrix}
        0&2&0&-1&0\\
        -1&0&0&0&0\\
        0&0&1&0&0\\
        0&1&0&2&3\\
        0&4&0&5&6
    \end{bmatrix}\)\begin{solution}
        We pick the column/row with the most zeros (in this case the second row or first column) then use the cofactor expansion. I'm going to omit the terms 
        that are being multiplied to 0 for my sanity.\begin{align*} 
            \begin{vmatrix}
                0&2&0&-1&0\\
                -1&0&0&0&0\\
                0&0&1&0&0\\
                0&1&0&2&3\\
                0&4&0&5&6
            \end{vmatrix} &= 1 \begin{vmatrix}
                2&0&-1&0 \\ 
                0&1&0&0 \\ 
                1&0&2&3 \\ 
                4&0&5&6
            \end{vmatrix} = 1 \begin{vmatrix}
                2&-1&0 \\ 
                1&2&3 \\ 
                4&5&6 
            \end{vmatrix} = 2 \begin{vmatrix}
                2&3 \\ 5&6 
            \end{vmatrix} + 1 \begin{vmatrix}
                1&3 \\ 4&6 
            \end{vmatrix} \\
            &= 2\,(12 - 15) + 1\,(6-12) = -6 -6 = -12. 
        \end{align*} The rows I picked in order were second row, then second row, then first row. If you're wondering why the sign changes at some points, remember the checkerboard
        of \(+-+\) that defines the cofactors. 
    \end{solution}
    \item \(\begin{bmatrix}
        1 &-1 &-1 &-1 &-1\\
        1 &2 &-1 &-1 &-1\\
        1 &2 &3 &-1 &-1\\
        1 &2 &3 &4 &-1\\
        1 &2 &3 &4 &5
    \end{bmatrix}\)
    \item \(\begin{bmatrix}
        0 &0 &0 &0 &5 &5\\
        0 &0 &3 &3 &3 &3\\
        0 &0 &0 &0 &0 &6\\
        1 &1 &1 &1 &1 &1\\
        0 &2 &2 &2 &2 &2\\
        0 &0 &0 &4 &4 &4
    \end{bmatrix}\)
    \item \(\begin{bmatrix}
        a & 0 & a \\ 0 & a & 0 \\ a & 0 & -a
    \end{bmatrix}\)
    \item \(\begin{bmatrix}
        b & b & c \\ 1 & -1 & 0 \\ 0 & 1 & 1
    \end{bmatrix}\)
    \item \(\begin{bmatrix}
        d & d & e & e \\ 1 & -1 & 1 & - 1 \\ 0 & 1 & -1 & 1 \\ 0 & 0 & 1 & -1
    \end{bmatrix}\)
    \item \(\begin{bmatrix}
        1 & 1 & 1 \\ r & s & t \\ r^2 & s^2 & t^2
    \end{bmatrix}\)
    \item \(\begin{bmatrix}
        1 -u & 2 \\ 3 & 4-u
    \end{bmatrix}\) 
    \item \(\begin{bmatrix}
        1-x & 1 & -1 \\ -1 & 2- x & 15 \\ 1 & -18 & 3 - x 
    \end{bmatrix}\)
    \item For items \#6 to \#11, find all real values (or tuples of real values) of the given variable/s that will make the determinant be equal to 0. 
    \item Find the determinants of the three elementary ``column'' operations, that is\begin{enumerate}
        \item Swapping two columns
        \item Multiplying all elements in a column by \(k\)
        \item Add a multiple of one column to another column
    \end{enumerate}
    \item Prove that any square matrix with a row of 0's or a column of 0's has determinant 0. 
    \item Show that any square matrix with two equal columns or two equal rows (i.e.\ having the same values) has determinant 0. 
    \item Prove that if the set of column vectors of a square matrix is linearly dependent, then the matrix has determinant 0. Also, prove the analogous result for row vectors. 
    \item Let \(A\) be an \(n\times n\) matrix such that \(A A^T = I_n \). Find all possible values for \(\text{det}(A)\). 
    \item Let \(A\) be a non-invertible \(n \times n \) matrix. Show that \(AB\) is also non-invertible, for any \(n\times n \) matrix \(B\). 
\end{enumerate}

\pagebreak
\section*{4.2: Eigenvectors, eigenvalues, and diagonalization}
\textit{My favorite lesson, because of course it is. This part is hard to grasp if you don't know the bigger picture, so I really recommend 
watching 3B1B's video on it to see what eigenvectors really are, geometrically.}

\begin{center}
    \colorbox{CornflowerBlue!50}{
        \begin{minipage}[c]{0.9\textwidth}
            \centering
            Determine whether or not the following vectors are eigenvectors for their corresponding matrix. If they are, give their corresponding eigenvalues. 
        \end{minipage}
    }\end{center}

\begin{enumerate}
    \setcounter{enumi}{18}
    \item \(\begin{bmatrix}
        3&0 \\ 8&-1 
    \end{bmatrix}\) and \(\begin{bmatrix}
        3 \\2 
    \end{bmatrix}\).\begin{solution}
        To check if a given vector is an eigenvector of some matrix, we simply check if it satisfies the equation \(A\vec{x} = \lambda\vec{x}\), where \(\lambda\) is 
        some real numer (i.e.\ not a vector). Then,\[
            \begin{bmatrix}
                3&0 \\ 8&-1 
            \end{bmatrix} \begin{bmatrix}
                3 \\2 
            \end{bmatrix} = \begin{bmatrix}
                9 \\ 22
            \end{bmatrix}.   
        \] This vector cannot be obtained by multiplying \(\begin{bmatrix}
            3\\2
        \end{bmatrix}\) by a real number, so \(\begin{bmatrix}
            3\\2
        \end{bmatrix}\) is not an eigenvector. 
    \end{solution}
    \item \(\begin{bmatrix}
        3&0 \\ 8&-1 
    \end{bmatrix}\) and \(\begin{bmatrix}
        2\\4
    \end{bmatrix}\).\begin{solution}
        Multiplying the matrix by the vector gives us\[
            \begin{bmatrix}
                3&0 \\ 8&-1 
            \end{bmatrix} \begin{bmatrix}
                2\\4 
            \end{bmatrix} = \begin{bmatrix}
                6 \\ 12
            \end{bmatrix} = 3\begin{bmatrix}
                2\\4
            \end{bmatrix}.   
        \] Thus, \(\begin{bmatrix}
            2\\4
        \end{bmatrix}\) is an eigenvector of the matrix, with eigenvalue \(\lambda=3\). 
    \end{solution}
    \item \(\begin{bmatrix}
        0&1&0 \\0&0&1 \\4&-17&8
    \end{bmatrix}\) and \(\begin{bmatrix}
        16\\4\\1
    \end{bmatrix}\).\begin{solution}
        Multiplying the matrix by the vector gives us\[
            \begin{bmatrix}
                0&1&0 \\ 0&0&1 \\ 4&-17&8
            \end{bmatrix} \begin{bmatrix}
                16\\4\\1
            \end{bmatrix} = \begin{bmatrix}
                4\\1\\4
            \end{bmatrix}. 
        \] This vector cannot be obtained by multiplying the original one by a real number, so it is not an eigenvector. 
    \end{solution}
    \item \(\begin{bmatrix}
        0&1&0 \\0&0&1 \\4&-17&8
    \end{bmatrix}\) and \(\begin{bmatrix}
        2-\sqrt{3}\\1\\2+\sqrt{3}
    \end{bmatrix}\).\begin{solution}
        Multiplying the matrix by the vector gives us\[
            \begin{bmatrix}
                0&1&0 \\ 0&0&1 \\ 4&-17&8
            \end{bmatrix} \begin{bmatrix}
                2-\sqrt{3}\\1\\2+\sqrt{3}
            \end{bmatrix} = \begin{bmatrix}
                1\\2+\sqrt{3}\\7+4\sqrt{3}
            \end{bmatrix} = \big(2+\sqrt{3}\big)\begin{bmatrix}
                2-\sqrt{3}\\1\\2+\sqrt{3}
            \end{bmatrix}.  
        \] Thus, the vector is an eigenvector of the matrix, with eigenvalue \(\lambda=2+\sqrt{3}\) If you don't believe that \({\bigl(2+\sqrt{3}\bigr)}^2 = 7 + 4\sqrt{3}\), that's 
        understandable, because like, where did that 7 come from? Expand it out yourself to verify. 
    \end{solution}
\end{enumerate}
\pagebreak
\begin{center}
    \colorbox{CornflowerBlue!50}{
        \begin{minipage}[c]{0.9\textwidth}
            \centering
            Find all \textit{real} eigenvalues of the following matrices. For each, describe the complete set of its corresponding eigenvectors. Also, determine which of the matrices can be diagonalized. 
            If they can be diagonalized, diagonalize them. If they cannot be diagonalized, explain why. 
        \end{minipage}
    }\end{center}

Before we proceed, I want to remark that \(\text{det}(\lambda I - A) = 0 \) is an equivalent statement to \(\text{det}(A - \lambda I) =0\). The handouts use \(\text{det}(\lambda I - A) = 0 \)
but I prefer \(\text{det}(A - \lambda I) =0\). This is because the former requires you to negate the matrix, \textit{then} add \(\lambda\) to the diagonals, whereas
the latter just requires you to subtract \(\lambda\) from the diagonals.\par 
For instance, let \(A=\begin{bmatrix}
    a&b \\ c&d
\end{bmatrix}\). Then, \(
    \lambda I - A = \begin{bmatrix}
        \lambda - a & - b \\ - c & \lambda - d
    \end{bmatrix}\) and \( A - \lambda I = \begin{bmatrix}
        a - \lambda & b \\ c & d - \lambda
    \end{bmatrix} 
\).\par 
Solving for both their determinants gives us \(\lambda^2 - ad\lambda + ad -bc\). 
I don't know about you guys, but I like less negative signs rather than more. But regardless, you can use either. 
\begin{enumerate}
    \setcounter{enumi}{22}
    \item \(\begin{bmatrix}
        1&4\\2&3 
    \end{bmatrix}\)\begin{solution}
        We begin by setting up the characteristic equation. We can do this by subtracting a variable, \(\lambda\) from the diagonal and setting the determinant of the resultant matrix to be 0. Then,\[
            \begin{vmatrix}
                1 - \lambda & 4 \\ 2 & 3-\lambda 
            \end{vmatrix}= (1-\lambda)\,(3-\lambda) - 8 = \lambda^2 - 4\lambda - 5 = (\lambda - 5)\,(\lambda + 1)\defeq 0 . 
        \] This gives us \(\lambda_1 = 5\) and \(\lambda_2 = -1\). Solving for their respective eigenvectors,\begin{itemize}
            \item For \(\lambda_1 = 5\), we need to solve \(\begin{bmatrix}
                -4 & 4 \\ 2 & -2 
            \end{bmatrix} \begin{bmatrix}
                x \\ y
            \end{bmatrix} = \begin{bmatrix}
                0\\0
            \end{bmatrix}\). Then, \(-4x + 4y = 0\), so \(x = y\).\par 
            Thus, the set of eigenvectors with eigenvalue \(\lambda=5\) is \(\begin{Bmatrix}
                \begin{bmatrix}
                    t \\ t
                \end{bmatrix} \Bigg|\, t \in \mathbb{R}
            \end{Bmatrix}\). 
            \item For \(\lambda_2 = -1\), we need to solve \(\begin{bmatrix}
                2 & 4 \\ 2 & 4 
            \end{bmatrix} \begin{bmatrix}
                x \\ y
            \end{bmatrix} = \begin{bmatrix}
                0\\0
            \end{bmatrix}\). Then, \(2x + 4y = 0\), so \(x = -2y\).\par 
            Thus, the set of eigenvectors with eigenvalue \(\lambda=-1\) is \(\begin{Bmatrix}
                \begin{bmatrix}
                    -2t \\ t
                \end{bmatrix} \Bigg|\, t \in \mathbb{R}
            \end{Bmatrix}\).  
        \end{itemize} For a matrix to be diagonalizable, the set of its eigenvectors must be linearly independent. Then, the diagonalization of \(A\) will be given by \(P D P^{-1}\), where \(P\) 
        is a matrix with the eigenvectors of \(A\) as its columns, and \(D\) is a matrix with the eigenvalues of \(A\) along its diagonal, each one in the same column as its corresponding eigenvector.\par 
        We can see that \(\begin{bmatrix}
            1\\1
        \end{bmatrix}\) is not a multiple of \(\begin{bmatrix}
            -2\\1
        \end{bmatrix}\), so the two vectors are linearly independent. This means our matrix is diagonalizable, and its diagonalization is given by \(\begin{bmatrix}
            1&4 \\ 2&3
        \end{bmatrix} = \dfrac{1}{3} \begin{bmatrix}
            1 &-2 \\ 1 &1
        \end{bmatrix}\begin{bmatrix}
            5 &0 \\ 0&-1 
        \end{bmatrix} \begin{bmatrix}
            1 & 2 \\ -1 & 1 
        \end{bmatrix}\). 
    \end{solution}
    \item \(\begin{bmatrix}
        1&2\\2&-1 
    \end{bmatrix}\)\begin{solution}
        We begin by setting up the characteristic equation. Then,\[
            \begin{vmatrix}
                1 - \lambda & 2 \\ 2 & -1-\lambda 
            \end{vmatrix}= (1-\lambda)\,(-1-\lambda) - 4 = \lambda^2 - 5 = (\lambda - \sqrt{5})\,(\lambda + \sqrt{5})\defeq 0 .     
        \] This gives us \(\lambda_1 = \sqrt{5}\) and \(\lambda_2 = -\sqrt{5}\). Solving for their respective eigenvectors,\begin{itemize}
            \item For \(\lambda_1 = \sqrt{5}\), we need to solve for \(x\) and \(y\) in the equation \(\begin{bmatrix}
                1-\sqrt{5} & 2 \\ 2 & -1-\sqrt{5} 
            \end{bmatrix} \begin{bmatrix}
                x \\ y
            \end{bmatrix} = \begin{bmatrix}
                0\\0
            \end{bmatrix}\).\par Then, we have \(x(1-\sqrt{5}) + 2y = 0\), so \(x = \dfrac{-2y}{1-\sqrt{5}}\).\par 
            Thus, the set of eigenvectors with eigenvalue \(\lambda=\sqrt{5}\) is \(\begin{Bmatrix}
                \begin{bmatrix}[1.2]
                    \frac{-2t}{1-\sqrt{5}} \\ t
                \end{bmatrix} \Bigg|\, t \in \mathbb{R}
            \end{Bmatrix}\). 
            \item For \(\lambda_1 = -\sqrt{5}\), we need to solve for \(x\) and \(y\) in the equation \(\begin{bmatrix}
                1+\sqrt{5} & 2 \\ 2 & -1+\sqrt{5} 
            \end{bmatrix} \begin{bmatrix}
                x \\ y
            \end{bmatrix} = \begin{bmatrix}
                0\\0
            \end{bmatrix}\).\par Then, we have \(x(1+\sqrt{5}) + 2y = 0\), so \(x = \dfrac{-2y}{1+\sqrt{5}}\).\par 
            Thus, the set of eigenvectors with eigenvalue \(\lambda=-\sqrt{5}\) is \(\begin{Bmatrix}
                \begin{bmatrix}[1.2]
                    \frac{-2t}{1+\sqrt{5}} \\ t
                \end{bmatrix} \Bigg|\, t \in \mathbb{R}
            \end{Bmatrix}\). 
        \end{itemize} We can see that \(\begin{bmatrix}[1.2]
            \frac{-2}{1-\sqrt{5}} \\ 1
        \end{bmatrix}\) and \(\begin{bmatrix}[1.2]
            \frac{-2}{1+\sqrt{5}} \\ 1
        \end{bmatrix}\) are not multiples of each other, so the two vectors are linearly independent.\par 
        Thus, its diagonalization is given by \(\begin{bmatrix}
            1&2\\2&-1 
        \end{bmatrix} = \dfrac{6}{-4\sqrt{5}} \begin{bmatrix}[1.2]
            \frac{-2}{1-\sqrt{5}} & \frac{-2}{1+\sqrt{5}} \\ 1&1
        \end{bmatrix}\begin{bmatrix}
            \sqrt{5} &0 \\ 0 & -\sqrt{5} 
        \end{bmatrix} \begin{bmatrix}[1.2]
            1 & \frac{2}{1+\sqrt5} \\ -1 &\frac{-2}{1-\sqrt{5}}
        \end{bmatrix}\). 
    \end{solution}
    \item \(\begin{bmatrix}
        3&5\\-1&-2 
    \end{bmatrix}\)
    \item \(\begin{bmatrix}
        3&0&0\\-2&7&0\\4&8&1 
    \end{bmatrix}\)
    \item \(\begin{bmatrix}
        9&-8&6&3\\0&-1&0&0\\0&0&3&0\\0&0&0&7 
    \end{bmatrix}\)
    \item \(\begin{bmatrix}
        4&0&-1\\0&3&0\\1&0&2 
    \end{bmatrix}\)
    \item \(\begin{bmatrix}
        3&1 \\ -1&5
    \end{bmatrix}\)
    \item \(\begin{bmatrix}
        1&0&1\\0&1&1\\0&0&2 
    \end{bmatrix}\)
    \item \(\begin{bmatrix}
        1&2&3\\0&0&-1\\0&1&0 
    \end{bmatrix}\)
    \item Let \(f_n\) be the \(n\)th entry of a second degree linear recurrence, where \(f_0 = s\) and \(f_1 = t \) are given constants. 
    Then, for \(n \geq 2\), let \(f_n = af_{n-1} + bf_{n-2}\), where again, \(a\) and \(b\) are given constants. Symbolically,\[
        f_n = \begin{cases}
            s & n=0 \\ 
            t & n=1 \\
            af_{n-1} + bf_{n-2} & n\geq 2
        \end{cases}. 
    \]\begin{enumerate}
        \item Let \(A = \begin{bmatrix}
            a&b \\1 & 0
        \end{bmatrix}\). Prove, by mathematical induction, that \(A^n \begin{bmatrix}
            f_1 \\ f_0 
        \end{bmatrix} = \begin{bmatrix}
            f_{n+1} \\ f_n
        \end{bmatrix}\).\begin{proof}
            Let the inductive hypothesis be that \(A^n \begin{bmatrix}
                f_1 \\ f_0 
            \end{bmatrix} = \begin{bmatrix}
                f_{n+1} \\ f_n
            \end{bmatrix}\), for all \(n \in \mathbb{N}\).\par 
            \textbf{Base Case.} When \(n=0\), \(
                A^0 \begin{bmatrix}
                    f_1 \\ f_0
                \end{bmatrix} = \begin{bmatrix}
                    1&0 \\ 0&1 
                \end{bmatrix}  \begin{bmatrix}
                    f_1 \\ f_0 
                \end{bmatrix} = \begin{bmatrix}
                    f_1 \\ f_0
                \end{bmatrix}
            \), which shows that it holds for the base case.\par
            \textbf{Inductive Step.} Assume that the statement holds for some value \(n\). Then,\begin{align*}
                A^{n+1} \begin{bmatrix}
                    f_1 \\ f_0 
                \end{bmatrix} &= A A^n \begin{bmatrix}
                    f_1 \\ f_0 
                \end{bmatrix} = A \begin{bmatrix}
                    f_{n+1} \\ f_n 
                \end{bmatrix} = \begin{bmatrix}
                    a &b \\ 1 &0 
                \end{bmatrix}\begin{bmatrix}
                    f_{n+1} \\ f_n 
                \end{bmatrix} = \begin{bmatrix}
                    af_{n+1} + bf_n \\ f_{n+1}
                \end{bmatrix} =\begin{bmatrix}
                    f_{n+2} \\ f_{n+1}
                \end{bmatrix},
            \end{align*} which shows that the statement holds for the inductive step. 
        \end{proof} 
        \item 
    \end{enumerate}
\end{enumerate}

\end{document}