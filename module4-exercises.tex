\documentclass{article}
\usepackage{amsmath}
\usepackage{amsthm,amssymb}
\usepackage[a4paper,left=25mm,right=25mm,top=30mm,bottom=30mm]{geometry}
\usepackage{fancyhdr}
\usepackage{titlesec}
\usepackage{enumerate} 
\usepackage{graphicx}
\usepackage[dvipsnames]{xcolor}
\usepackage{tikz} 
\usepackage{cancel}
\usepackage{parskip}
\usepackage[condensed,light,math]{iwona}
\usepackage[T1]{fontenc}
\usepackage{booktabs}

\title{probability exercises}
\author{emilianna louise abundo limlengco} 
\date{\today} 

\fboxsep=4pt
\RenewDocumentCommand{\footnoterule}{}{\vfill\kern-3pt \hrule width 0.4\columnwidth\kern2.6pt} %yoinked from LSE

\RenewDocumentCommand{\labelitemi}{}{$\rightarrow$}
\RenewDocumentCommand{\labelenumi}{}{\colorbox{pink}{\textbf{\arabic{enumi}}}}
\RenewDocumentCommand{\labelenumii}{}{\colorbox{CornflowerBlue!50}{\textbf{\alph{enumii}}}}

\NewDocumentEnvironment{solution}{}{%
    \RenewDocumentCommand{\qedsymbol}{}{$\blacksquare$}
    \begin{proof}[Solution]
}{\end{proof}}

\makeatletter
\newcommand*{\defeq}{\mathrel{\rlap{%
                     \raisebox{0.3ex}{$\m@th\cdot$}}%
                     \raisebox{-0.3ex}{$\m@th\cdot$}}%
                     =}
\makeatother

\begin{document} 

\section*{How to Use this Reviewer}
Hello! This is a compilation of solved exercises for module 4 of MATH 51.4. All of these exercises are taken straight from Aldrich and Cisco's course notes, so you can expect tests 
to be very similar to the items given. Some items are bound to be a little bit trickier than others, so I'll note when these items show up.\par Normal items will look like this:\begin{enumerate} 
    \item A very easy math problem. What's 1 + 1?
\end{enumerate} 
whereas difficult problems will be soulless, like this:\begin{enumerate}\setcounter{enumi}{1}
    \RenewDocumentCommand{\labelenumi}{}{\fcolorbox{magenta}{white}{\textbf{\arabic{enumi}}}}
    \item A very difficult math problem. Prove that $\displaystyle \binom{2n}{n} < 2^{2n-2},~\forall n \geq 5$ using induction. 
\end{enumerate} I might also include warnings in my \textbf{Nerd Interjections!}\par
\parindent=25pt \begin{minipage}[t]{.14\textwidth}
    \vspace{0pt}
    \includegraphics[width=2cm]{nerd_maddy.png}
\end{minipage}%
\fbox{
\begin{minipage}[t]{.76\textwidth}
    \vspace{0pt}
    \textbf{Nerd Interjection!}\footnote{Image from @Ellem\_\_ on Twitter.} These sections are for me to remind you of some necessary information to solve the problems, elaborate on 
    something that I think isn't all that clear with just pure math symbols, give a helpful theorem, be an annoying piece of shit, anything, really! Just think of it as a tips and tricks section. 
\end{minipage}%
}\parindent=0pt \par I also have another section called \textbf{Can we Prove it?} (unfortunately lacking a cute picture to go along with it; Mikh 
was nice enough to edit one up for me, but I haven't been able to format it in a way I like), where I include some interesting, not really necessary, but 
nonetheless relevant proofs. So far, these two are my only two gimmicks, but I might add more in the future.\par
\fbox{\begin{minipage}[t]{0.98\textwidth}
    \vspace{0pt} 
    \textbf{Can we Prove it?} This is just a random proof I yoinked from our homeworks.\begin{proof} 
        ($ \implies $) Let $ x \in (A \cap B) \setminus C $. Then, $ x \in (A \cap B)$ and $ x \notin C $. \\
        \phantom{($ \implies $)} Since $x \in (A \cap B)$, $ x \in A$ and $ x \in B$. \\
        \phantom{($ \implies $)} Since $x \in A$ and $x \notin C$, $x \in (A \setminus C) $. \\
        \phantom{($ \implies $)} Since $x \in B$ and $x \notin C$, $x \in (B \setminus C) $. \\
        \phantom{($ \implies $)} Thus, $x \in (A \setminus C) \cap (B \setminus C) $. \\ 
        \\
        ($ \impliedby $) Let $ x \in (A \setminus C) \cap (B \setminus C) $. Then, $ x \in (A \setminus C) $ and $ x \in (B \setminus C) $. \\ 
        \phantom{($ \impliedby $)} Since $ x \in (A \setminus C) $, $ x \in A $ and $ x \notin C $. \\
        \phantom{($ \impliedby $)} Since $ x \in (B \setminus C) $, $ x \in B $ and $ x \notin C $. \\
        \phantom{($ \impliedby $)} Since $ x \in A $ and $ x \in B $, $ x \in (A \cap B) $. \\
        \phantom{($ \impliedby $)} Thus, $ x \in (A \cap B) \setminus C $. \\
        \\ 
        Since both sides of the conditional are true, it holds that $ (A \cap B) \setminus C = (A \setminus C) \cap (B \setminus C) $. 
    \end{proof} 
\end{minipage}%
}\par
Finally, there are blue boxes to indicate when instructions aren't obvious from the question itself, or if there are similar items that can be grouped together.\par
\parindent=25pt 
    \colorbox{CornflowerBlue!50}{
    \begin{minipage}[c]{0.9\textwidth}
        \centering
        For items \#7 to \#12, we need to reevaluate our life decisions.
    \end{minipage}
    }\parindent=0pt \par 
It's very important to note that this is a \textit{work in progress!} I am human, and I will make mistakes, and I cannot finish doing all the exercises within the span of one day. If you spot anything wrong, 
please feel free to message me; I will correct it as soon as possible.\par
As a final note, these are not replacements for the modules/paying attention in class, these are supplements for them. I won't explain all the topics here, and I'll assume that you at least have 
read the basics, so don't treat these reviewers as your only source of information. Our teachers spends a lot of time on the handouts, they're really good! (except when they're wrong) With that, though, I think 
I've covered all pertinent points. Good luck, and happy studying!
\pagebreak 

\end{document}